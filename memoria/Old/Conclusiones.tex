\chapter{Conclusiones}\label{cap:conclusiones}
A lo largo de este Trabajo Fin de Grado se ha expuesto una propuesta educativa completa, junto con el desarrollo de una plataforma para ejecutarla, debiendo ser esta plataforma de fácil acceso e instalación. Se han explicado los pasos dados para cumplir los objetivos marcados para esta plataforma y su puesta en práctica basada en ejercicios prácticos, con suficiente detalle como para poder reproducirlos sin ayuda.
\par Realizaremos a continuación un resumen de los objetivos iniciales y cómo se han resuelto.
\section{Conclusiones}
Teníamos dos objetivos principales en este Trabajo Fin de Grado:
\begin{enumerate}
	\item \textbf{Plataforma PyBo-Kids-2.0.} Se ha desarrollado un \textit{middleware} que capacita a alumnos y alumnas a programar el robot mBot en el lenguaje Python, abstrayendo la dificultad de la sintaxis de este lenguaje y del Arduino nativo de la placa base del robot, y permitiendo utilizar una biblioteca de funciones que ocultan la lógica más compleja y deja libertad para el aprendizaje, tanto de robótica como de programación. \\
	Esta plataforma, llamada PyBo-Kids-2.0 por ser una versión de un trabajo anterior de JdeRobot (ver en \cite{JdeRobot}), cuenta con un programa ''residente'' que se ejecuta en el robot de forma que el estudiante no deba programar nada en Arduino, y una biblioteca Python que, importándose como un módulo, facilita la programación del robot.\\
	Es posible utilizar PyBo-kids-2.0 en un Sistema Operativo Windows, por ser el más extendido y el más habitual en hogares y centros educativos, y requiriendo solamente la instalación de Python, en caso de que el curso escolar se imparta de forma controlada (en caso de que se quisiera utilizar de forma individual, tan solo requiere de la instalación adicional de Arduino), siendo ésta una ventaja de cara a utilizarla en cursos de robótica educativa. 
	
	\item \textbf{Propuesta educativa.} Los ejercicios propuestos están diseñados cumpliendo unos objetivos docentes individuales y de forma colectiva, siendo ésta última la iniciación a la robótica y a la programación orientada a Educación Primaria; responden a una experiencia práctica y se ha podido comprobar su efectividad y viabilidad. Para cada ejercicio se ha explicado la metodología y los pasos a seguir para guiar al estudiante en su resolución. Se ha mantenido presente la situación y necesidades educativa de un alumno de edad temprana, y se han adaptado los contenidos a ellos.  
\end{enumerate}
\section{Trabajos futuros}
La plataforma diseñada cumple con los objetivos que nos habíamos marcado para este Trabajo Fin de Grado, siendo utilizable en un entorno real. Sin embargo, hay líneas y posibilidades en las que mejorar o ampliarlo. A continuación listaremos posibles líneas de continuación de la plataforma:
\begin{itemize}
	\item Añadido de sensores y actuadores del mBot: el estado actual de la plataforma contempla el mBot básico, aunque existen más periféricos que añadir, y con los que poder ampliar tanto las bibliotecas como los contenidos educativos. Lo siguientes ejemplos son sensores u actudores que se pueden conectar al mBot: una matriz LED en la que es posible ''dibujar'', un receptor IR con el que poder usar un mando a distancia con el que dirigir al robot, un sensor de humedad, un sensor de sonido, y más.
	\item Adaptar el sistema para enviar como parámetro al residente Arduino el puerto en el que conectarlos, haciendo posible arrancar sólo los sensores deseados desde el programa principal. Es decir, iniciar los sensores desde el lado PC. Esta posibilidad mejoraría el rendimiento y velocidad del programa, ya que no Arduino no ejecutaría todos los sensores a la vez.
	\item Adaptar el sistema de mensajes para hacer posible conectar dos sensores del mismo tipo. La implementación de esta opción tiene como requerimiento la anterior, ya que se utilizarían dos puertos para los sensores, por lo que se hace necesario inicializar el sensor desde el lado PC.
	\item Mejorar el rendimiento de las bibliotecas con el uso de hilos de ejecución, con el fin de tener dos hilos diferentes leyendo y escribiendo del canal Serial.
	\item Añadir la funcionalidad de Bluethoot para que las bibliotecas Python y Arduino reconozcan el \href{https://www.makeblock.es/productos/adaptador_bluetooth_usb/}{adaptador de bluethooth de Makeblock}, que crea un puerto serie una vez emparejado al robot. Esta funcionalidad es posible con Scratch, la cual se ha utilizado en algunas de las prácticas, de forma nativa. Esto ampliará el número de ejercicios posibles a realizar, ya que no tendremos la limitación física de un cable conectado al PC.
\end{itemize} 
Cada cambio o ampliación requerirá de cambios en las dos bibliotecas, para que la funcionalidad de cara al estudiante no cambie:
\begin{itemize}
	\item En caso de añadir un actuador o un sensor: añadir los métodos de lectura y parseo del mensaje o de composición y escritura del mensaje  en el lado Residente, con su correspondiente identificador numérico; añadir igualmente los métodos de lectura o de escritura en el lado PC.
	\item En caso de cambios en la forma de leer los periféricos, o de hilos de ejecución o cambios en el protocolo de mensajes, los cambios oportunos deben realizarse en las dos bibliotecas.
	\item Añadido, o cambio, de las funciones ''visibles'' en el lado PC, que serán las que puedan ser utilizadas por los alumno y alumnas para programar el robot. 
	\item Añadido de las posibles nuevas funciones en la ayuda visible de la biblioteca PC.
	\item Compilar y subir el nuevo programa residente al robot, para poder ser utilizado de forma independiente con el programa Python.
	\item Comprobación de que el sistema Residente-PC funciona, con el nuevo Residente y los cambios realizados. 
	\item Actualizar la biblioteca mBot.py en el repositorio donde se esté trabajando con los estudiantes. 
\end{itemize}