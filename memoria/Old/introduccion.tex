\chapter{Introducción}\label{cap:introduccion}

\section{Contexto}\label{sec:contextoIntroduccion}
Durante mucho tiempo la robótica ha sido una disciplina muy alejada y poco accesible al público, tanto por su coste económico, como por el difícil acceso a recursos académicos (manuales teóricos o recursos humanos). Esto ha convertido, en lo que se refiere a una visión popular, a la robótica en una disciplina de lujo, accesible a unos pocos privilegiados económica o académicamente: NASA, grandes corporaciones, robots complejos y super funcionales, etc, llegando incluso a ser más cercana a la ciencia ficción que a la ingeniería corriente. No era vista como una disciplina científica cercana a la que poder dedicarse, mucho menos como una asignatura accesible a estudiantes jóvenes o universitarios. \\
Además, el ''resultado'' de la robótica, tampoco era visto como algo real, sino como ciencia ficción o productos de lujo económico: no se concebían los robot como algo adaptable a la vida corriente ni se pensaba en soluciones \textit{robóticas} a problemas reales. A esta visión de la robótica y de los robots ha contribuido en gran medida la literatura y el cine, creando en el imaginario popular robot humanoides indistinguibles de personas reales y completamente independientes y funcionales, inteligencias artificiales utilizadas para viajes espaciales, o que crecen y se desarrollan al margen de la humanidad.\\

Durante los últimos años, no obstante, se ha observado un cambio de paradigma, impulsado desde el mundo de la ingeniería y el pragmatismo comercial. Los robots han dejado de verse como ''humanos electrónicos'', convirtiéndose poco a poco en una definición más realista: proceso electrónico programado para cumplir una o varias funciones interactuando con el medio. Tenemos varios ejemplos de esta nueva definición de robótica:

\begin{description}
	\item [Automatismos.] También llamados \textit{\textbf{bots}}, son un robot sin una capa física, un proceso que funciona de forma independiente recogiendo datos, procesándolos y respondiendo a ellos de forma \textit{inteligente}. Tenemos muchos ejemplos de bots, de los primeros que se le puede ocurrir a cualquier persona el conocido ''chat de contacto'' en las aplicaciones web (asistentes de ayuda al cliente), aunque también están ampliamente extendidos en aplicativos de \textit{metadata}. Otro uso muy extendido de estos automatismos son los procesos de seguridad de los vehículos modernos, que se mantienen recogiendo datos del medio (de velocidad, proximidad y velocidad de otros vehículos, distancias entre carriles, etc) para ofrecer seguridad añadida frente a posibles errores humanos.\\
	Una de las características principales de estos \textit{procesos robotizados} es que arrancan sin requerir interacción -aparte de que luego respondan a peticiones o estímulos humanos, no requieren de que se les ''ordene'' encenderse. Siempre pueden apagarse, obviamente por motivos de seguridad, pero volverán a arrancar con la misma configuración y, lo más importante, sin necesitar de ningún reset o configuración inicial.
	\item [Internet Of Things.]\footnote{Internet de las cosas} Muy escuchado, aunque popularmente no se conecte con la robótica, la idea básica del IOT es añadirle conexión a Internet a objetos de uso diario, con los objetivos de:
	\begin{itemize}
		\item Añadirle funcionalidades que requieren de conexión online, por ejemplo un reloj inteligente, con el que poder atender mensajes o llamadas o poder pagar compras con NFC. Con estas funcionalidades, la persona interactúa para obtener una respuesta, sin embargo se mantienen funcionando siempre. Es decir, funciona esperando un estímulo, y responde a él.
		\item Recoger datos para procesarlos y ofrecer información añadida en función de esos datos. Por ejemplo, con el mismo reloj inteligente, que recoge datos de sueño, y te muestra la calidad y horas de éste. Con estas funciones no es necesario interactuar, sino que estás preparadas para recoger datos siempre que estén disponibles, y mostrar al usuario los resultados de los procesos que tenga programados.
	\end{itemize}
	El IOT funciona con procesos robotizados, programados para recoger datos de forma automática, o responder a estímulos del medio, que siguen las mismas normas antes mencionadas: se mantienen funcionando (siempre que no se les apague expresamente) sin necesidad de configuraciones añadidas, sino que su programación se encarga de ello. Ciertamente, muchos objetos con IOT necesitan de datos iniciales, pero esto es debido a que los datos recogidos son biométricos y necesitan de un contexto (de unos parámetros) para el procesamiento de datos. 
	\item [Domótica.] Posiblemente la utilidad robótica más extendida a día de hoy. Cada vez más unida al IOT, aunque no lo requiere necesariamente. Viene del significado latino de 'casa' (\textit{domus}) y engloba a todo sistema o automatismo diseñado para cumplir una función que facilite una tarea doméstica, o le añada funcionalidad a esa tarea. Tenemos sistemas de seguridad, aspiradores, electrodomésticos que se comportan diferente dependiendo de las circunstancias (cantidad de agua de una lavadora, o temperatura de un frigorífico), control de luces con aplicaciones móviles (tanto con Internet como sin él), controles de accesos, etc. Todo esto son procesos robotizados, robots al fin y al cabo, que utilizan información del medio para reaccionar de una u otra manera, ya sea información de sensores (la lavadora o el frigorífico), o información que reciba del usuario/a (apagar o encender las luces, la TV o la alarma).
	
	\item [Inteligencia Artificial.] Uno de los grandes trabajos respecto al cambio de visión de la robótica, ha sido conseguir cambiar el ideario de Inteligencia Artificial, y de separarlo del concepto de robot. Un robot puede contener un cierto grado de IA, pero no es necesario para el concepto de robótica. Una Inteligencia Artificial es un proceso robótico que, además de responder al medio, se adapta a él. Es decir, se realimenta con la información que va recogiendo del medio para poder ofrecer respuestas cada vez más precisas. Lejos del concepto que tradicionalmente se tenía de Inteligencia Artificial (como hemos comentando, en parte culpa de la ciencia ficción, por ejemplo, el robot M.A.D.R.E. de la pelicula \textit{Alien}), podemos encontrar inteligencias artificiales habitualmente en nuestro día a día. Los asistentes virtuales de diferentes fabricantes (Google, Amazon o Apple), aprenden del medio en mayor o menor medida, y son capaces de buscar una respuesta que no tienen pre-programada; los programadas de procesamiento de datos masivos (\textit{big data}) se retro-alimentan con los datos que recogen y sus resultado para afinar esos mismos algoritmos de procesamiento.
	
	\item [Juguetes robots] Finalmente, llegamos a la aplicación más reconocible como ''robot'', también muy lucrativa y, en cuanto a este Trabajo Fin de Grado, la que más nos interesa. La robótica ha tenido un gran impulso en el ocio, creándose diferentes juguetes (tanto para niños como para adultos) y convirtiéndolos de un juguete de élite, a tener diversos robots, con diversas funcionalidades, y accesibles a más variedad de costes. Esta accesibilidad de la robótica en el ocio ha contribuido en gran medida a esa transformación de la robótica y a incluirla en nuestras vidas.
\end{description}


Estos cambios por parte de la industria y avances por parte de la ingeniería han contribuido a acercar la robótica a las personas de a pie, sobre todo a las nuevas generaciones, facilitando el acceso a recursos con los que aprender y creando un interés por ello. Es muy notable el abaratamiento de los diversos componentes, así como de los productos finales, y la introducción de esta disciplina en los diferentes niveles educativos (sobre todo en estudios universitarios): que más gente tenga acceso a los recursos necesarios, y tenga desde joven los conceptos inculcados, contribuye a aumentar las aplicaciones prácticas en todos los ámbitos, y a resolver problemas en los que anteriormente, al no ser una disciplina extendida, quizá nadie había reparado. \\

Sin embargo, y a pesar de todo este trabajo realizado y todo lo conseguido, aún queda mucho por trabajar. El principal problema a la hora de que una persona se ponga a aprender cómo programar un \textit{comportamiento robótico} es que debe aprenderlo desde cero sin haber tenido un aprendizaje progresivo como con el resto de enseñanzas. La Matemática se enseña con niveles progresivos de dificultad, desde edades muy tempranas con las metodologías y conceptos ampliamente discutidos y formulados, para poder llegar a las Matemáticas más avanzadas. Igualmente ocurre con la Física, Química, o cualquier otra disciplina. Nadie espera que una persona sin conocimiento ninguno de estas materias, arranque  un curso de, por ejemplo, Cálculo Diferencial, Matemática Discreta o Teoría Gravitatoria. Es decir, se \textbf{educa} al cerebro desde el principio, cuando más capacidad de aprendizaje se tiene, para adaptarse a nuevos conceptos y poder aprenderlos gradualmente. \\
\par Con la robótica y el pensamiento computacional, no se ha seguido ese orden lógico. La mayoría de los estudiantes que aprenden Programación, Computación o Algoritmia lo hacen a edades más avanzadas, ya sea por su cuenta o con educación reglada. Si esta educación comenzara, como otras disciplinas, con los alumnos y alumnas siendo mucho más jóvenes y avanzara con ellos durante las siguientes etapas educativas, se conseguiría que llegaran a las más avanzadas mucho más preparados y, sobre todo, despertaría en muchos más estudiantes la curiosidad por estas carreras, tanto profesional como educacionalmente.

\section{Propósito principal}
Con la idea de introducir en la robótica a niños y jóvenes como principal foco de atención, el propósito de este Trabajo Fin de Grado es ofrecer una fórmula para la enseñanza de robótica y programación a alumnos y alumnas sin conocimientos previos, tanto de Educación Primaria como Secundaria. Para ello se ha utilizado un robot educativo, llamado mBot, del fabricante Makeblock (ver en \cite{makeblock}), que cumple los requisitos de simpleza y accesibilidad que requiere el trabajar con niños y niñas. Además, está especialmente preparado para la enseñanza sin dejar de ser un \textit{juguete}, ya que buscamos despertar el interés y convencer de la facilidad de una disciplina a menudo vista como especialmente complicada y aburrida.

Este proyecto estará compuesto de dos partes. Una, la creación de una manera sencilla de programar el mBot en el lenguaje de programación Python, sin ser éste el nativo del robot, pero mucho más sencillo. Se explicará el proceso seguido para desarrollar esta solución, poniendo énfasis en detallar los pasos necesarios para replicarlo, utilizarlo en clases reales con alumnos e incluso ampliarlo para un alumno o alumna o en caso de querer continuar con el proyecto educacional. En la segunda parte encontraremos una propuesta completa de uso práctico de esta plataforma, con ejercicios detallados y soluciones de referencia, aumentando el nivel de dificultad progresivamente y con objetivos conceptuales claramente marcados.

Con estas dos partes, obtendremos una forma de enseñar conceptos como Programación, Algoritmia o metodologías de desarrollo de Software de forma accesible y fácil, sin entrar en explicaciones farragosas y poco adaptadas a edades alrededor de los 7-15 años.


\section{Antecedentes}\label{sec:antecedentes}

Dentro de la Universidad Rey Juan Carlos, el proyecto JdeRobot y particularmente dentro de él, el proyecto PyBoKids (Ver en \citen{JdeRobot}) han servido de antecedente ideológico a este Trabajo Fin de Grado. Ofrecían recursos para robótica educativa, teniendo en cuenta diferentes robot, diferentes plataformas y orientado a varios rangos de edades pre-universitarias.\\

En este Trabajo Fin de Grado hemos tomado como ejemplo el propósito educativo, y el concepto de \textit{middleware entre el estudiante y el robot} que se propone en PyBoKids. Basándonos en la experiencia de un curso escolar de clases de robótica con alumnos de Educación Primaria, hemos tenido la necesidad de una solución educativa sencilla, lo más accesible posible en cuanto a herramientas y conceptos, y con objetivos marcados y preparados para que cualquiera pueda utilizarlos.


