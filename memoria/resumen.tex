
\chapter*{Resumen}


\label{cap:resumen}

Este Trabajo Fin de Grado se fundamenta en una necesidad de enseñar robótica y pensamiento computacional a alumnos y alumnas sin conocimientos previos ni de robótica ni de programación. El objetivo es ofrecer una solución para introducir e interesar a estudiantes en la disciplina, de forma sencilla y efectiva, evitando las dificultades propias de la programación de robots.\\

Con esta finalidad, ofrecemos una plataforma para la programación del robot mBot en Python, creando una capa de aplicación visible al usuario que ocultará toda la funcionalidad de las  comunicaciones con el robot real, que sólo admite programación en Arduino. La característica principal de esta plataforma será la accesibilidad, en herramientas tanto software como hardware, con el objetivo de hacerla lo más accesible posible a la mayor cantidad de estudiantes.\\

Hemos diseñado un curso escolar, fundamentalmente práctico, estructurado en dos partes. Con la primera, utilizamos la plataforma del mBot ofrecida por el fabricante, de programación visual por bloques. Durante la segunda parte del curso utilizaremos la plataforma y API desarrollada en Python, con un lenguaje de texto más cercano a la programación clásica.\\

\afterpage{\null\newpage}